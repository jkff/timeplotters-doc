\documentclass{beamer}

\usepackage{hyperref}

\title{\mbox{How my visualization tools use little memory:}\\
\mbox{A tale of incrementalization and laziness}}
\author[Eugene Kirpichov]{Eugene Kirpichov $<$ekirpichov@gmail.com$>$}
\date{Nov 7, 2012}
\institute[SPb fprog]{St.-Petersburg Functional Programming User Group}

\begin{document}

\begin{frame}
\titlepage
\end{frame}

\begin{frame}
\frametitle{Outline}
\tableofcontents
\end{frame}

\section{Introduction}

\begin{frame}
  \frametitle{Introduction}
  \url{http://jkff.info/software/timeplotters}

  Tools for visualizing program behavior from logs, optimized for one-liners.

  \begin{itemize}
    \item \textbf{timeplot}~--- quantitative graphs about multiple event streams.
    \begin{itemize}
      \item Histograms of event durations, of types of events, regular line/dot plots etc.
    \end{itemize}
    \item \textbf{splot}~--- Gantt-like chart about a large number of concurrent processes.
    \begin{itemize}
      \item Birds-eye view of thread/process/machine interaction patterns
    \end{itemize}
  \end{itemize}
\end{frame}

\begin{frame}
  \frametitle{Live demo}

  \begin{center}
  Live demo
  \end{center}
\end{frame}

\begin{frame}
  \frametitle{Why optimize}
  Because they were slow and a memory hog (boxing, thunks).
  \begin{itemize}
    \item Took minutes for 100,000's of events
    \item Took hours or crashed for 1,000,000's of events
    \begin{itemize}
    \item Crashing was the last straw. I couldn't do my job.
    \end{itemize}
  \end{itemize}
\end{frame}

\begin{frame}
  \frametitle{How to optimize}
  Rewrite in C or Java? We can do better!
  \begin{itemize}
    \item It would be a defeat :)
    \item Would rewrite and re-debug everything
    \item Would learn nothing
  \end{itemize}
  Turned out worth it.
\end{frame}

\section{splot}

\begin{frame}
  \frametitle{splot}
  The easy one.

  \begin{columns}[t]
    \column{.5\textwidth}
    \begin{block}{Before}
    \begin{enumerate}
      \item Read input into a list
      \item Traverse to calibrate axes
      \item Traverse to render
    \end{enumerate}
    \end{block}

    \column{.5\textwidth}
    \begin{block}{After}
    \begin{enumerate}
      \item Read input into a list
      \item Traverse to calibrate axes
      \item Read input into a list
      \item Traverse to render
    \end{enumerate}
    Lazy IO does the rest.
    \end{block}
  \end{columns}
  \vskip10px
  Cost: no output to window, no input from stdin.

  Code tour.
\end{frame}

\section{tplot}

\begin{frame}
  \frametitle{tplot: why it can NOT be done}
  The hard one.

  \begin{itemize}
    \item Complex data flow: events:tracks = M:M.
    \item Uses Chart, which keeps data in memory and for good reasons.
  \end{itemize}

  Code tour: old version.
\end{frame}

\begin{frame}
  \frametitle{tplot: why it CAN be done}

  Reading too much $\neq$ drawing too much $\Rightarrow$ Chart is not a problem.

  \vskip10px

  Building ``data to render'' for Chart in 1 pass seemed \emph{possible}.

  \vskip10px

  Code tour: PlotData, Render.hs
\end{frame}

\begin{frame}
  \frametitle{tplot: main idea}
  
  Push-based:
  \begin{itemize}
    \item List representation unchanged and hidden
    \item Push item
    \item Get result (at any moment)
  \end{itemize}

  \vskip10px

  \emph{Code tour: StreamSummary.
  
  Live example: average + profiling. 
  
  Applicative average.}
\end{frame}

\section{Types of stream operations}

\begin{frame}
  \frametitle{Types of stream operations}

  \begin{center}
  \begin{tabular}{ccc}
    Concept & Logical type & Actual type \\
    \hline
    Summarize & \texttt{[a] $\to$ b}  & Summary a b \\
    Transform & \texttt{[a] $\to$ [b]} & Summary b r $\to$ Summary a r \\
    Generate & \texttt{a $\to$ [b]} & Summary b r $\to$ (a $\to$ r) \\
  \end{tabular}
  \end{center}

  \vskip10px

  Composition pipelines become quite funny.

  \vskip10px

  \emph{Code tour: RLE etc.}
\end{frame}

\begin{frame}
  \frametitle{tplot: new architecture}
  New architecture:
  \begin{itemize}
    \item Push-based builders for all plot types
    \item Driver loop to feed input events to output tracks
    \item When done, summarize and render
  \end{itemize}
  \emph{Code tour: driver loop, plots (vs old code).}
\end{frame}

\section{The transition}

\begin{frame}
  \frametitle{tplot: the transition}
  I wanted to keep things working all the time.

  \begin{enumerate}
    \item Change interface~--- make the change possible.
    \item Change implementation~--- make the change real.
  \end{enumerate}
\end{frame}

\begin{frame}
  \frametitle{tplot: the transition}
  Before: Completely non-incremental.
  \begin{itemize}
    \item \emph{(interface)} Separate building and rendering
    \item Split into modules
    \item \emph{(interface)} Explicit 2 passes, both \emph{potentially} incremental
    \item Toying with incremental combinators to get a feeling for them
    \item \emph{(interface)} Make all plots have an incremental interface \\
    (but actually use \texttt{toList} bridge)
    \item \emph{(implementation)} Incrementalize plots one by one
  \end{itemize}
  After: Completely incremental.
\end{frame}

\section{Optimization}

\begin{frame}
  \frametitle{Not so easy}
  Memory leaks due to insufficient strictness:\\
  ``push isn't pushing hard enough.''

  \begin{center}
  \begin{block}{Question}
  What and why remains unevaluated until too late?
  \end{block}
  \end{center}

  \textbf{Profilers} didn't help at all. \textbf{Debug.Trace} is insufficient: it outputs lines and I need \emph{hierarchy}, not sequence.
\end{frame}

\begin{frame}
  \frametitle{Enter Debug.HTrace}

  \texttt{cabal install htrace}

  ``Code'' ``tour''. Live demo: average.
\end{frame}

\begin{frame}
  \frametitle{Why not X?}

  Iteratee, conduit, pipes, $\ldots$

  \begin{itemize}
    \item Scary, so I tried to get as far as I can without them
    \item Ended up very small, simple and nice, so I didn't look back
    \item Also educational
  \end{itemize}
\end{frame}

\section{Lessons learnt}

\begin{frame}
  \frametitle{Lessons learnt}

  \begin{itemize}
    \item Lazy IO is ok for this task
    \item Debugging laziness is hard: I wouldn't start a high-risk real-time project in Haskell
    \item Profilers are almost useless for debugging laziness
    \item Debug.Trace is better, Debug.HTrace much better
    \item Push-based incremental processing is fun and easy
  \end{itemize}
\end{frame}

\end{document}
